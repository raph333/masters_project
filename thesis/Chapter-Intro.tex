\chapter{Introduction}
\label{chapter:Introduction}

\section{Predicting molecular properties}

\textit{} Briefly the applications of computational chemistry and computational structural biology

\section{Graph neural networks}

{\itshape
\noindent Explain the reasoning behind graph neural networks:

\begin{itemize}
	\item no order => permutation invariance required
	\item no fixed dimension: a node can have 0 to n neighbors => need aggregation function
\end{itemize}

\noindent Nomenclature:\\
1) graph conv: complicated but compares well with computer vision\\
2) message passing: simple, but not analogous to cv
}

\section{Limitations of Graph neural networks for molecular structures}

\subsection{Lack of 3D structure representation}

{\itshape
	
subsubsection: different structural arrangements can lead to same graph => same results

what does not work:\\
* adding coordinates: not invariant to translations and rotations\\
* adding relative positions as edge-features: invariant to translations but not to rotations
=> opposite case to above: now to equivalent structures can be represented with different edge-features

subsubsection: no downsampling

this goes hand in hand with: no graph-level or graph-part-level features; only node-features - even when higher level

compare with computer vision

see: results
	}
	
	


