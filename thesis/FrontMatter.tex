% Use Roman numerals (i, ii, iii, etc.) for page numbers in the front matter.
\pagenumbering{roman}

%%%%%%%%%%%%%%%%%%%%%%%%%%%%%%%%%%%%%%%%%%%%%%%%%%%%%%%%%%%%%%%%%
%% TITLE PAGE.
%%%%%%%%%%%%%%%%%%%%%%%%%%%%%%%%%%%%%%%%%%%%%%%%%%%%%%%%%%%%%%%%%

% No headers or footers on the title page.
\thispagestyle{empty}

\begingroup
\centering
\setstretch{1.0}
~
\\[1em]
\sffamily\bfseries\fontsize{26}{31.2}\selectfont
\DocumentTitle
%\\
%Use Manual Line Breaks If Necessary
\\[0.4in]
\normalfont\large
%Thesis by
%\\[0.25em]
%\sffamily\bfseries
\Large{\AuthorName}
\vspace{1cm}\\
\normalsize{supervised by}\\
\Large{Roland Kwitt, PhD}
%\\[0.4in]
%\normalfont\normalsize
%In Partial Fulfillment of the Requirements
%\\[0.5em]
%for the Degree of
%\\[0.5em]
%Doctor of Philosophy
%\\[0.5em]
%in
%\\[0.5em]
%Electrical Engineering and Computer Science
\vfill
%\includegraphics[height=1.8in]{Figure-SchoolLogo}
%\\[1.5em]
University of Salzburg
\\[0.5em]
Salzburg, Austria
\\[1.5em]
2020
\par
\endgroup

\clearpage

%%%%%%%%%%%%%%%%%%%%%%%%%%%%%%%%%%%%%%%%%%%%%%%%%%%%%%%%%%%%%%%%%
%% COPYRIGHT PAGE.
%%%%%%%%%%%%%%%%%%%%%%%%%%%%%%%%%%%%%%%%%%%%%%%%%%%%%%%%%%%%%%%%%

\pagestyle{plain}
\setcounter{page}{2}

\begingroup
\centering
\setstretch{1.0}
\null
\vfill
{\sffamily\textcopyright}~2020
\\[0.5em]
\AuthorName
\\[0.5em]
All Rights Reserved
\par
\endgroup

\clearpage

%%%%%%%%%%%%%%%%%%%%%%%%%%%%%%%%%%%%%%%%%%%%%%%%%%%%%%%%%%%%%%%%%
%% DEDICATION PAGE.
%%%%%%%%%%%%%%%%%%%%%%%%%%%%%%%%%%%%%%%%%%%%%%%%%%%%%%%%%%%%%%%%%

%\begingroup
%\centering
%\setstretch{1.0}
%~
%\\[1in]
%\textit{Insert dedication here}
%\par
%\endgroup
%
%\clearpage

%%%%%%%%%%%%%%%%%%%%%%%%%%%%%%%%%%%%%%%%%%%%%%%%%%%%%%%%%%%%%%%%%
%% ACKNOWLEDGMENTS.
%%%%%%%%%%%%%%%%%%%%%%%%%%%%%%%%%%%%%%%%%%%%%%%%%%%%%%%%%%%%%%%%%

%\chapter*{Acknowledgments}
%\addcontentsline{toc}{chapter}{Acknowledgments}
%
%Insert thesis acknowledgments here
%
%\clearpage

%%%%%%%%%%%%%%%%%%%%%%%%%%%%%%%%%%%%%%%%%%%%%%%%%%%%%%%%%%%%%%%%%
%% ABSTRACT.
%%%%%%%%%%%%%%%%%%%%%%%%%%%%%%%%%%%%%%%%%%%%%%%%%%%%%%%%%%%%%%%%%

\section*{\Huge{Abstract}}
\addcontentsline{toc}{chapter}{Abstract}

\vspace{1cm}
\noindent This thesis investigates the application of deep neural networks to predict molecular properties from structures of small organic molecules. The raw data consists of atom coordinates in 3D space, while the target variables are either atomic interactions (Section~\ref{sec:champs}) or properties of molecules (Section~\ref{sec:alchemy}). Two machine learning competitions provided the datasets as well as the opportunity to benchmark the results against other machine learning practitioners.

While numerous approaches can be chosen to tackle this challenge, the project at hand uses the strategy of representing molecules as graphs and training deep neural networks to predict the target variables. The type of neural networks that takes graph-structured data as input is called graph neural networks in general, but most of the models follow an architecture described as Message Passing Neural Networks (MPNNs)~\cite{Gilmer2017} or Graph Convolutional Neural Networks~\cite{Schutt2017}. This model class has interesting commonalities with regular convolutional networks, but also differs from them in important aspects. This thesis provides an overview of this model class and presents an example model-architecture. The investigation indicates that the representation of the 3D geometry (which is lacking in regular MPNNs) is one of the key obstacles when applying MPNNs to molecular structures. Several changes in model architecture to address this shortcoming are proposed and evaluated (Section~\ref{sec:root-node} and~\ref{sec:direction-vectors}). Moreover, the results provide evidence that the pure deep learning approach of using raw data only as model input (which is common practice in computer vision and natural language processing) is superior to manual feature engineering in chemical applications as well (Section~\ref{sec:raw-data}). Finally, the experiments also show that the choice of input data representation can be more important than the choice of model architecture (Section~\ref{sec:neighborhood-expansion}).

In summary, this work shows some interesting challenges when using MPNNs in chemical applications and presents results that point towards potential improvements. 


\clearpage

%%%%%%%%%%%%%%%%%%%%%%%%%%%%%%%%%%%%%%%%%%%%%%%%%%%%%%%%%%%%%%%%%
%% TABLE OF CONTENTS (TOC), LISTS OF FIGURES, TABLES, ETC.
%%%%%%%%%%%%%%%%%%%%%%%%%%%%%%%%%%%%%%%%%%%%%%%%%%%%%%%%%%%%%%%%%

\tableofcontents

%\listoffigures
%\listoftables

\clearpage

% Use Arabic numerals (1, 2, 3, etc.) for subsequent page numbers.
\pagenumbering{arabic}
